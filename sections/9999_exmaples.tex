% !TEX root = ../main.tex
\newpage
\section{Template examples \label{sec:9999_ex}}
Here you can see how to use the theorem environments. Find more of them in the style.sty file

\subsection{Examples \label{sec:fermat}}

Usually a chapter in mathematics does not start with a theorem but with a defintiion. For example

\begin{dfn}[Continous functions] \label{dfn:continous}
A function $f: D_f \rightarrow \R$ is called \textbf{continuous}  at $x_o \in D_f$ if \par
\begin{equation*}
    \forall \, \epsilon > 0 \quad  \exists \, \delta > 0 \quad  \forall \, x \in D_f : \ \abs{x-x_0} < \delta \  \Rightarrow  \ \abs{f(x) - f(x_0)} < \epsilon
\end{equation*}
\end{dfn}

With the label tag you can referenence the defintion of a continous function \ref{dfn:continous} later on in the manuscript which is a nice feature. The following parts are from the original author of the template showing some more features.

There are many famous theorems in Mathematics. One of the most famous theorems is Fermat's Last Theorem.

\begin{thm}[Fermat's Last Theorem] \label{thm:fermat}
If $n>2$, there are no integers $a,b,c$ with $abc \neq 0$ such that $a^n + b^n= c^n$.
\end{thm}


% Traditional Mathematics
\subsection{More examples}

Theorem~\ref{thm:fermat} is one of the most famous theorems in Mathematics. But most undergraduate students do not learn Fermat's Last Theorem. Instead, many students learn formulas such as:
	\[
	\oint_C \mathbf{F} \cdot d\mathbf{r} = \iint_S \nabla \times \mathbf{F} \cdot dS
	\]
But Mathematics starts far more simple than that. The first topic in Mathematics that one typically sees is Arithmetic. For instance, students typically will learn ``FOIL.'' 
	\begin{equation} \label{eq:foil}
	(x + y)^2= x^2 + 2xy + y^2
	\end{equation}
However, \eqref{eq:foil} tends to be a stumbling block for students. Many students will instead claim, incorrectly, that $(x+y)^2= x^2 + y^2$. The purpose of this course will be to prove Dirichlet's Unit Theorem, which states:

\begin{restatable*}[Dirichlet's Unit Theorem]{thm}{unit} \label{thm:unit}
Let $K$ be a number field of degree $n$ with $r$ real embeddings and $s$ conjugate pairs of complex embeddings. Then the abelian group $\O_K^\times$ is a finitely generated abelian group with rank $r+s-1$ and $\O_K^\times \cong \mu_K \times \Z^{r+s-1}$, where $\mu_K$ are the roots of unity in $\O_K$. 
\end{restatable*}


However, it will take some time to prove Theorem~\ref{thm:unit}.

\subsection{Even more examples}
Recall that the goal of this course was to prove Dirichlet's Unit Theorem:

\unit	% Dirichlet's Unit Theorem

\pf L.T.R. \qed


\begin{ex}
If $K=\Q$, then $r=1$ and $s=0$ so that $r+s-1=0$. Therefore, $\O_\Q^\times=\Z^\times=\{\pm1\}$. Of course, this is the most trivial possible example. \xqed
\end{ex}
