% !TEX root = ../main.tex
\newpage
\section{Basic Mathematical Background: Introduction}

\subsection{Basic Definitions}

\begin{dfn}[Group]\label{dfn:11_group}
Let $G \neq \emptyset $ be a nonempty set and $\circ: \; G \times G \rightarrow G$ a  binary operation. $(G, \circ)$ is a \textit{group} if the following conditions are full filled:
\begin{enumerate}
    \item neutral element: $\exists \, e\in G \; \; \forall \, g \in G: \; e \circ g = g \circ e = g$
    \item inverse element: $\forall \, g \in G \; \; \exists \, g^{-1} \in G: \; g^{-1} \circ g = g \circ g^{-1} = e$
    \item associative law: $\forall a,b,c \in G: \; \left(a \circ b \right) \circ c = a \circ \left(b \circ c \right)$
\end{enumerate}

\noindent G is \textit{Abelian} if elements commute: $\forall a,b \in G: \; a \circ b = v \circ a$
\end{dfn}

\begin{dfn}[Subgroup]\label{dfn:12_subgroup}
Let $(G, \circ)$ be a group and $U$ a set with $U \subset G$. $(U, \circ)$ is a \textit{subgroup} of $(G, \circ)$ if
\begin{enumerate}
    \item $(U, \circ)$ is a group by definition \ref{dfn:11_group}
    \item $(U, \circ)$ is closed: $\forall \, u,v \in U: u \circ v \in U$
\end{enumerate}
\end{dfn}

Often the notation for a group $(G, \circ)$ is shortened to $G$ if the operation is clear from the context or does not need to be specified.

\begin{dfn}[Order and period]\label{dfn:13_order}
Let $(G, \circ)$ be a group. 
\begin{itemize}
    \item \textit{order of the group} $G$: $\# G = \abs{G}$
    \item \textit{order $n$ of the element} $g \in G$: $g^n = e$ \par in finite groups the order of each element is finite
    \item \textit{period of the element} $g \in G$: $\langle g \rangle = \lbrace e, g, g^2, \dots, g^{n-1} \rbrace$ \par $\langle g \rangle$ is an abelian subgroup of $G$.
\end{itemize}
\end{dfn}

\begin{thm}[Rearrangement Theorem]\label{thm:11_rearrangement}
Let $G = \lbrace e, g_1, g_2, \dots g_n \rbrace$ be a group. \par
\noindent For any $g_i \in G: \; g_i \circ G = \lbrace g_i \circ e, g_i \circ g_1, \dots g_i \circ g_n \rbrace$ contains each element og $G$ once and only once.
\end{thm}

A consequence of the rearrangement theorem is that each coloumn of a multiplication table contains each element only once.

\subsection{Cosets and Classes}

\begin{dfn}[Coset]\label{dfn:14_coset}
Let $G$ be a group and $U \subseteq G$ a subgroup of $G$. \par
\noindent For $x \in G: \; Ux = \lbrace e\circ x, u_1\circ x, \dots u_r \circ x \rbrace$ is the \textit{right coset} of $U$
\end{dfn}

The same rational leads to the definition of the \textit{left coset} as $xU$. A coset is not necessarily a subgroup but if $x \in U$ then $Ux = U$ by the rearrangement theorem. 

\begin{thm}[Distinct cosets]\label{thm:12_cosets-dist}
Let $Ux, Uy$ be right cosets of a subgroup $U \subseteq G$. Then either $Ux \cap Uy = \emptyset$ or $Ux = Uy$. 
\end{thm}

\begin{thm}[Group order divisor]\label{thm:13_order-div}
The order of a subgroup is a divisor of the group order. 
\end{thm}

\subsection{(Self-)Conjugacy}

\begin{dfn}[Conjugate]\label{dfn:15_conjugate}
Let $G$ be a group and $a,b \in G$. \par 
\noindent $b$ \textit{conjugate} $a \quad \Leftrightarrow \quad \exists \, x \in G: \; b = xax^{-1}$ 
\end{dfn}

\begin{thm}\label{thm:14_conj-eqv}
Conjugacy is an equivalence relation:
\begin{enumerate}
	\item Reflexivity: $a$ conj. $a$
	\item Symmetric: $a$ conj. $b \quad \Rightarrow \quad b$ conj. $a$
	\item Transitivity: $a$ conj. $b \wedge b$ conj. $c \quad \Rightarrow \quad a$ conj. $c$
\end{enumerate}
\end{thm}

\begin{dfn}[Class]\label{dfn:15_class}
A \textit{class} is the totality of elements which can be obtained from a group element by conjugation.
\end{dfn}

\begin{thm}[Class order]\label{thm:15_class-order}
All elements of the same class have the same order. 
\end{thm}

\begin{dfn}[self-conjugate subgroup]\label{dfn:16_self-conj}
Let $N \subseteq G$ be a subgroup of group $G$. \par
$N$  is self-conjugate$\quad \Leftrightarrow \quad \forall \, x \in G: xNx^{-1} = N$ 
\end{dfn}
This means a subgroup is self-conjugate if it is invariant under conjugation with all elements of the group. A self-conjugate group is sometimes called \textit{invariant, normal or normal divisor} 

\begin{dfn}[Simple group]\label{dfn:17_simple-gr}
A group with no self-conjugate subgroup is \textit{simple}.
\end{dfn}

\begin{thm}\label{thm:16_lr-cosets}
The right and left cosets of the self-conjugate subgroup $N \subseteq G$ are the same: $xN = Nx, \; x \in G$. 
\end{thm}




